\documentclass[
                ]
{article}

    \usepackage{lmodern}


\usepackage{amssymb,amsmath}
\usepackage{ifxetex,ifluatex}
\usepackage{fixltx2e} % provides \textsubscript

\ifnum 0\ifxetex 1\fi\ifluatex 1\fi=0 % if pdftex

    \usepackage[T1]{fontenc}
    \usepackage[utf8]{inputenc}

    
\else % if luatex or xelatex

    \ifxetex
        \usepackage{mathspec}
        \usepackage{xltxtra,xunicode}
    \else
        \usepackage{fontspec}
    \fi
    \defaultfontfeatures{Mapping=tex-text,Scale=MatchLowercase}
    \newcommand{\euro}{€}

    
    
    
    
\fi


% use upquote if available, for straight quotes in verbatim environments
\IfFileExists{upquote.sty}{\usepackage{upquote}}{}

% use microtype if available
\IfFileExists{microtype.sty}{
    \usepackage{microtype}
    \UseMicrotypeSet[protrusion]{basicmath} % disable protrusion for tt fonts
}{}


    \usepackage[
                    margin=1in    ]
    {geometry}












    \usepackage{color}
    \usepackage{fancyvrb}
    \newcommand{\VerbBar}{|}
    \newcommand{\VERB}{\Verb[commandchars=\\\{\}]}
    \DefineVerbatimEnvironment{Highlighting}{Verbatim}{commandchars=\\\{\}}
    % Add ',fontsize=\small' for more characters per line
    \usepackage{framed}
    \definecolor{shadecolor}{RGB}{248,248,248}
    \newenvironment{Shaded}{\begin{snugshade}}{\end{snugshade}}
    \newcommand{\AlertTok}[1]{\textcolor[rgb]{0.94,0.16,0.16}{#1}}
    \newcommand{\AnnotationTok}[1]{\textcolor[rgb]{0.56,0.35,0.01}{\textbf{\textit{#1}}}}
    \newcommand{\AttributeTok}[1]{\textcolor[rgb]{0.77,0.63,0.00}{#1}}
    \newcommand{\BaseNTok}[1]{\textcolor[rgb]{0.00,0.00,0.81}{#1}}
    \newcommand{\BuiltInTok}[1]{#1}
    \newcommand{\CharTok}[1]{\textcolor[rgb]{0.31,0.60,0.02}{#1}}
    \newcommand{\CommentTok}[1]{\textcolor[rgb]{0.56,0.35,0.01}{\textit{#1}}}
    \newcommand{\CommentVarTok}[1]{\textcolor[rgb]{0.56,0.35,0.01}{\textbf{\textit{#1}}}}
    \newcommand{\ConstantTok}[1]{\textcolor[rgb]{0.00,0.00,0.00}{#1}}
    \newcommand{\ControlFlowTok}[1]{\textcolor[rgb]{0.13,0.29,0.53}{\textbf{#1}}}
    \newcommand{\DataTypeTok}[1]{\textcolor[rgb]{0.13,0.29,0.53}{#1}}
    \newcommand{\DecValTok}[1]{\textcolor[rgb]{0.00,0.00,0.81}{#1}}
    \newcommand{\DocumentationTok}[1]{\textcolor[rgb]{0.56,0.35,0.01}{\textbf{\textit{#1}}}}
    \newcommand{\ErrorTok}[1]{\textcolor[rgb]{0.64,0.00,0.00}{\textbf{#1}}}
    \newcommand{\ExtensionTok}[1]{#1}
    \newcommand{\FloatTok}[1]{\textcolor[rgb]{0.00,0.00,0.81}{#1}}
    \newcommand{\FunctionTok}[1]{\textcolor[rgb]{0.00,0.00,0.00}{#1}}
    \newcommand{\ImportTok}[1]{#1}
    \newcommand{\InformationTok}[1]{\textcolor[rgb]{0.56,0.35,0.01}{\textbf{\textit{#1}}}}
    \newcommand{\KeywordTok}[1]{\textcolor[rgb]{0.13,0.29,0.53}{\textbf{#1}}}
    \newcommand{\NormalTok}[1]{#1}
    \newcommand{\OperatorTok}[1]{\textcolor[rgb]{0.81,0.36,0.00}{\textbf{#1}}}
    \newcommand{\OtherTok}[1]{\textcolor[rgb]{0.56,0.35,0.01}{#1}}
    \newcommand{\PreprocessorTok}[1]{\textcolor[rgb]{0.56,0.35,0.01}{\textit{#1}}}
    \newcommand{\RegionMarkerTok}[1]{#1}
    \newcommand{\SpecialCharTok}[1]{\textcolor[rgb]{0.00,0.00,0.00}{#1}}
    \newcommand{\SpecialStringTok}[1]{\textcolor[rgb]{0.31,0.60,0.02}{#1}}
    \newcommand{\StringTok}[1]{\textcolor[rgb]{0.31,0.60,0.02}{#1}}
    \newcommand{\VariableTok}[1]{\textcolor[rgb]{0.00,0.00,0.00}{#1}}
    \newcommand{\VerbatimStringTok}[1]{\textcolor[rgb]{0.31,0.60,0.02}{#1}}
    \newcommand{\WarningTok}[1]{\textcolor[rgb]{0.56,0.35,0.01}{\textbf{\textit{#1}}}}



    \usepackage{longtable,booktabs}


    \usepackage{graphicx}
    \makeatletter

    \def\maxwidth{
        \ifdim
            \Gin@nat@width>\linewidth\linewidth
        \else
            \Gin@nat@width
        \fi}

    \def\maxheight{
        \ifdim
            \Gin@nat@height>\textheight\textheight
        \else
            \Gin@nat@height
        \fi}

    \makeatother

    % Scale images if necessary, so that they will not overflow the page
    % margins by default, and it is still possible to overwrite the defaults
    % using explicit options in \includegraphics[width, height, ...]{}
    \setkeys{Gin}{width=\maxwidth,height=\maxheight,keepaspectratio}


\ifxetex
    \usepackage[setpagesize=false, % page size defined by xetex
                            unicode=false, % unicode breaks when used with xetex
                            xetex]{hyperref}
    \else
        \usepackage[unicode=true]{hyperref}
    \fi
    \hypersetup{breaklinks=true,
                            bookmarks=true,
                            pdfauthor={Chris Conlan (First Author); Another Author (Second Author)},
                            pdftitle={Test PDF Rendered from Math Markdown},
                            colorlinks=true,
                            citecolor=blue,
                            urlcolor=blue,
                            linkcolor=magenta,
                            pdfborder={0 0 0}}

    \urlstyle{same}  % don't use monospace font for urls

    



\setlength{\parindent}{0pt}
\setlength{\parskip}{6pt plus 2pt minus 1pt}
\setlength{\emergencystretch}{3em}  % prevent overfull lines


    \setcounter{secnumdepth}{0}




%%% Use protect on footnotes to avoid problems with footnotes in titles
\let\rmarkdownfootnote\footnote%
\def\footnote{\protect\rmarkdownfootnote}

%%% Change title format to be more compact
\usepackage{titling}

% Create subtitle command for use in maketitle
\newcommand{\subtitle}[1]{
    \posttitle{
        \begin{center}\large#1\end{center}
        }
}

\setlength{\droptitle}{-2em}
    \title{Test PDF Rendered from Math Markdown}
    \pretitle{
        \vspace{
            \droptitle
        }
        \centering\huge
    }
    \posttitle{\par}


    \subtitle{A Subtitle}


    \author{
                    Chris Conlan (First Author)                    Another Author (Second Author)    }
    \preauthor{\centering\large\emph}
    \postauthor{\par}


    \predate{\centering\large\emph}
    \postdate{\par}
    \date{March 9, 2018}



%%%
% Begin the actual document where the writing goes ...
%%%
\begin{document}

%%% Run the previously defined custom routine
\maketitle



%%% Worth learning how to activate the following three commands
    {
        \hypersetup{linkcolor=black}
        \setcounter{tocdepth}{3}
        \tableofcontents
    }


    \listoftables

    \listoffigures

%%%
% This seems to be incredibly important
%%%
\hypertarget{header-a1}{%
\section{Header A1}\label{header-a1}}

\hypertarget{header-a2}{%
\subsection{Header A2}\label{header-a2}}

\hypertarget{header-a3}{%
\subsubsection{Header A3}\label{header-a3}}

\hypertarget{header-a4}{%
\paragraph{Header A4}\label{header-a4}}

\hypertarget{header-a5}{%
\subparagraph{Header A5}\label{header-a5}}

Header A6

\hypertarget{header-b1}{%
\section{Header B1}\label{header-b1}}

\hypertarget{header-b2}{%
\subsection{Header B2}\label{header-b2}}

\hypertarget{header-b3}{%
\subsubsection{Header B3}\label{header-b3}}

\hypertarget{header-b4}{%
\paragraph{Header B4}\label{header-b4}}

\hypertarget{header-b5}{%
\subparagraph{Header B5}\label{header-b5}}

Header B6

\url{http://example.com/auto-linked-url}

Some text\ldots{}

\textbf{bold}

\emph{slanty}

\texttt{monoscript}

\textbf{\texttt{bold\ monoscript}}

\begin{Shaded}
\begin{Highlighting}[]
\CommentTok{# Python highlighted syntax}
\ImportTok{import}\NormalTok{ numpy }\ImportTok{as}\NormalTok{ np}
\BuiltInTok{print}\NormalTok{(np.arange(}\DecValTok{1}\NormalTok{,}\DecValTok{10}\NormalTok{) }\OperatorTok{*} \DecValTok{3}\NormalTok{)}

\CommentTok{"""}
\CommentTok{Some nice python docstrings}
\CommentTok{"""}
\end{Highlighting}
\end{Shaded}

A regression\ldots{}

\[
\hat{y}_i = \hat{\beta}_0 + \hat{\beta}_1 x_{i,1} + \hat{\beta}_2 x_{i,2} + ... + \hat{\beta}_n x_{i,n}
\]

Now some \(\alpha\) inline \(\beta\) math \(\gamma\) with \(\delta\) lot
\(\eta\) of \(\zeta\) Greek.

\begin{figure}
\centering
\includegraphics{./jazz_dog.jpeg}
\caption{Jazz Dog}
\end{figure}

\begin{longtable}[]{@{}clrl@{}}
\caption{A fancy multiline table}\tabularnewline
\toprule
\begin{minipage}[b]{0.15\columnwidth}\centering
Centered Header\strut
\end{minipage} & \begin{minipage}[b]{0.10\columnwidth}\raggedright
Default Aligned\strut
\end{minipage} & \begin{minipage}[b]{0.20\columnwidth}\raggedleft
Right Aligned\strut
\end{minipage} & \begin{minipage}[b]{0.32\columnwidth}\raggedright
Left Aligned\strut
\end{minipage}\tabularnewline
\midrule
\endfirsthead
\toprule
\begin{minipage}[b]{0.15\columnwidth}\centering
Centered Header\strut
\end{minipage} & \begin{minipage}[b]{0.10\columnwidth}\raggedright
Default Aligned\strut
\end{minipage} & \begin{minipage}[b]{0.20\columnwidth}\raggedleft
Right Aligned\strut
\end{minipage} & \begin{minipage}[b]{0.32\columnwidth}\raggedright
Left Aligned\strut
\end{minipage}\tabularnewline
\midrule
\endhead
\begin{minipage}[t]{0.15\columnwidth}\centering
First\strut
\end{minipage} & \begin{minipage}[t]{0.10\columnwidth}\raggedright
row\strut
\end{minipage} & \begin{minipage}[t]{0.20\columnwidth}\raggedleft
12.0\strut
\end{minipage} & \begin{minipage}[t]{0.32\columnwidth}\raggedright
Example of a row that spans multiple lines.\strut
\end{minipage}\tabularnewline
\begin{minipage}[t]{0.15\columnwidth}\centering
Second\strut
\end{minipage} & \begin{minipage}[t]{0.10\columnwidth}\raggedright
row\strut
\end{minipage} & \begin{minipage}[t]{0.20\columnwidth}\raggedleft
5.0\strut
\end{minipage} & \begin{minipage}[t]{0.32\columnwidth}\raggedright
Here's another one. Note the blank line between rows.\strut
\end{minipage}\tabularnewline
\bottomrule
\end{longtable}





\end{document}
